\section{Conclusions and final thoughts}
\label{sec:conclusions_and_final_thoughts}

\subsection{Summary of Contributions}
In this thesis, we have presented a comprehensive study on orienting hyperedges in the context of metabolistic pathways. 
Our investigation focused on the elaboration of a AMPL model that would allow the a researcher to obtain a good solution by using ILP optimization. 
The major contributions of our work include:

\begin{itemize}
    \item The creation of a logically robust model that can be used to solve the problem of orienting hyperedges in a metabolistic pathway.
    \item The introduction of additional constraints believed to have some significance in the field of metabolistic pathways. These include:
    \begin{enumerate}
        \item Preventing the inversion of certain reactions.
        \item Marking certain vertices as internal.
        \item Marking certain vertices as external.
    \end{enumerate}
    \item We have benchmarked the model with respect to alternative models, established its strengths and weaknesses and discussed when it is most appropriate to apply it.
\end{itemize}


\subsection{Limitations and Challenges}
While our research has provided valuable insights, it is important to acknowledge the limitations and challenges encountered during the study. These limitations include:

\begin{itemize}
    \item Currently, there is no proof that the problem is P or NP. \\
    Although we strongly suspect that the problem is NP, if it is discovered that it is P, there may exist a polynomial-time algorithm that finds the optimal set of solutions.
    \item As mentioned in Section \ref{sec:stakeholders}, there currently isn't a consensus on whether our results bear any biological significance.
\end{itemize}

\subsection{Future Research Directions}
Building on the findings of this thesis, there are several promising avenues for future research. These may include:

\begin{itemize}
    \item To determine whether the results of our model can be used to predict the direction of reactions in a metabolistic pathway.
    \item The creation of a tool to better way to visualize hyper-graphs.
    \item To analyse the problem of maximizing the number of internal vertices of a hyper-graph to determine whether it is NP-hard.
\end{itemize}

\subsection{Final Thoughts}

In concluding this thesis, we have embarked on a journey to tackle the complex challenge of orienting hyperedges in metabolic pathways. 
Our endeavor has yielded a robust AMPL model that stands as a valuable tool for researchers grappling with the intricacies of metabolic network analysis.

The quest to understand the biological significance of our results remains an ongoing endeavor. The ambiguity surrounding the P or NP nature of the problem underscores the intricate nature of metabolic pathway orientation. This uncertainty fuels our curiosity and motivates us to delve deeper into the underlying complexities.

I would like to express my sincere gratitude to Professor Gabriel Valiente Feruglio for offering me the invaluable opportunity to study this fascinating topic. Their guidance, encouragement, and unwavering support have played a pivotal role in shaping my academic journey.

As we close this chapter, it is our hope that this work sparks curiosity and inspires future researchers to continue unraveling the mysteries within metabolic networks.





